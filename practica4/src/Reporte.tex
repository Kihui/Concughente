%Especificacion
\documentclass[12pt]{article}

%Paquetes
\usepackage[left=2cm,right=2cm,top=3cm,bottom=3cm,letterpaper]{geometry}
\usepackage{lmodern}
\usepackage[T1]{fontenc}
\usepackage[utf8]{inputenc}
\usepackage[spanish,activeacute]{babel}
\usepackage{mathtools}
\usepackage{amssymb}
\usepackage{enumerate}
\usepackage{float}
\usepackage{enumitem}
\usepackage{hyperref}
%\usepackage{tabularx}
%\usepackage{wasysym}
\usepackage{graphicx}
%\graphicspath { {media/} }
%\usepackage{pifont}

%Preambulo
\title{Computación Concurrente \\ Práctica 4: Reporte}
\author{Andrea Itzel González Vargas\\ andreagonz@ciencias.unam.mx \\ 108000801\\Carlos Gerardo Acosta Hernández\\carlos-acosta@ciencias.unam.mx\\3-11154243}
\date{Facultad de Ciencias UNAM}

\begin{document}
\maketitle
\section{Preguntas}
\begin{enumerate}[label=(\alph*)]
\item \textbf{¿Qué tipo de paso de mensajes se realiza en la clase \textit{RemoteMessagePassing}?}\\
  La comunicación en \textit{RemoteMessagePassing} es punto a punto, bidireccional y asíncrona, pues \textit{JAVA} cuenta en su
  implementación de \textit{sockets} y \textit{streams} con un \textit{buffer} de información de entrada, por lo que en realidad los emisores no se bloquean\footnote{Esto de hecho es observable en la implementación y su ejecución.} en espera de que el receptor haya recibido su mensaje.
 \item \textbf{Si N threads (T1, T2, ..., TN) envían, cada uno, un mensaje diferente a S, en el orden de su numeración (primero T1, luego T2, etc.), ¿En qué orden va a recibir S los mensajes?}\\
   Los sockets que utilizamos son del tipo \textit{TCP}\footnote{\href{http://www.buyya.com/java/Chapter13.pdf}{Fuente: }, págs. 350-355}, además de que es por ello que la comunicación es bidireccional, esto asegura la entrega de los mensajes y el orden de llegada debe ser el mismo que el orden de envío. Deben recibirse entonces en el orden T1, T2, ..., TN.
\end{enumerate}

\section{Ejecución}
\subsection{QuickSort}
Para la ejecución de la implementación del \textit{QuickSort} con \textit{Rendezvous}, basta
compilar el código fuente pertinente:
\begin{verbatim}
    $ javac QuickSort
\end{verbatim}
Posteriormente, y similarmente a los \textit{tests} de las otras dos clases de los ejercicios previos
a éste, se puede iniciar un servidor haciendo:
\begin{verbatim}
    $ java QuickSort server
\end{verbatim}
Que estará esperando la conexión con algún cliente que enviará a su vez un arreglo a ordenar por el
servidor esperando como respuesta uno ya ordenado.
Para iniciar tal cliente, es necesario llamar al programa con dos argumentos (se espera que ya exista un servidor en ejecución).
\begin{verbatim}
    $ java QuickSort client <cantidad_de_numeros_a_ordenar>
\end{verbatim}
Cabe mencionar que nuestro cliente sólo hace una petición por ejecución mientras que el servidor se mantiene en ejecución en espera de mensajes para hacer su trabajo de ordenamiento.

\subsection{RemoteMessagePassing y ExtendedRendezvous}
Las pruebas respectivas de ambos \textit{tests} se ejecutan\footnote{Aunque en el caso de nuestro \textit{RemoteMessagePassingTest} la comunicación se basa en que el cliente manda por tiempo indefinido mensajes, mientras que en \textit{ExtendedRendezvousTest} la comunicación es de ambas partes, tanto del cliente como del servidor.} como se especifica en la práctica.


\end{document}
