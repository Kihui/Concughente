%Especificacion
\documentclass[12pt]{article}

%Paquetes
\usepackage[left=2cm,right=2cm,top=3cm,bottom=3cm,letterpaper]{geometry}
\usepackage{lmodern}
\usepackage[T1]{fontenc}
\usepackage[utf8]{inputenc}
\usepackage[spanish,activeacute]{babel}
\usepackage{mathtools}
\usepackage{amssymb}
\usepackage{enumerate}
\usepackage{float}
\usepackage{enumitem}
%\usepackage{tabularx}
%\usepackage{wasysym}
\usepackage{graphicx}
%\graphicspath { {media/} }
%\usepackage{pifont}

%Preambulo
\title{Computación Concurrente \\ Práctica 4: Reporte}
\author{Andrea González Vargas\\Carlos Gerardo Acosta Hernández}
\date{Facultad de Ciencias UNAM}

\begin{document}
\maketitle
\section{Preguntas}
\begin{enumerate}[label=(\alph*)]
\item \textbf{¿Qué tipo de paso de mensajes se realiza en la clase \textit{RemoteMessagePassing}?}\\
  
\item \textbf{Si N threads (T1, T2, ..., TN) envían, cada uno, un mensaje diferente a S, en el orden de su numeración (primero T1, luego T2, etc.), ¿En qué orden va a recibir S los mensajes?}\\
  
\end{enumerate}

\section{Ejecución}
\subsection{QuickSort}
Para la ejecución de la implementación del \textit{QuickSort} con \textit{Rendezvous}, basta
compilar el código fuente pertinente:
\begin{verbatim}
    $ javac QuickSort
\end{verbatim}
Posteriormente, y similarmente a los \textit{tests} de las otras dos clases de los ejercicios previos
a éste, se puede iniciar un servidor haciendo:
\begin{verbatim}
    $ java QuickSort server
\end{verbatim}
Que estará esperando la conexión con algún cliente que enviará a su vez un arreglo a ordenar por el
servidor esperando como respuesta uno ya ordenado.
Para iniciar tal cliente, es necesario llamar al programa con dos argumentos (se espera que ya exista un servidor en ejecución).
\begin{verbatim}
    $ java QuickSort client <cantidad_de_numeros_a_ordenar>
\end{verbatim}
Cabe mencionar que nuestro cliente sólo hace una petición por ejecución mientras que el servidor se mantiene en ejecución en espera de mensajes para hacer su trabajo de ordenamiento.

\section{RemoteMessagePassing y ExtendedRendesvous}
Las pruebas respectivas de ambos \textit{tests} se ejecutan\footnote{Aunque en el caso de \textit{RemoteMessagePassing} la comunicación se basa en que el cliente manda por tiempo indefinido mensajes, mientras que en \textit{ExtendedRendezvous} la comunicación es de ambas partes, tanto del cliente como del servidor.} como se especifica en la práctica.


\end{document}
